% !TEX root = thesis.tex

\usepackage[utf8]{inputenc}
\usepackage[T1]{fontenc} % better fonts

\usepackage[dutch,english]{babel} % put english last to make it default!
\usepackage{csquotes}

% If you need chinese characters, uncomment this:
% \usepackage{CJKutf8}
% \newcommand{\mychinese}[1]{\begin{CJK*}{UTF8}{gbsn}#1\end{CJK*}}

\usepackage{graphicx,soul,latexsym,mathrsfs,graphics,stmaryrd}
\usepackage{amsfonts,amsmath,amssymb,wasysym,float,booktabs,url}
\usepackage{microtype,etex,enumerate,setspace,xcolor,mdframed}
\usepackage{epigraph,caption,bussproofs,tabto}

% NOTE add your own \usepackage commands here

\usepackage[section]{placeins}

\usepackage{tikz}
\usetikzlibrary{backgrounds,positioning,trees,shapes,arrows,patterns,topaths,calc}

\usepackage[
    backend=biber,
    style=alphabetic,
    natbib=true,
    url=true,
    maxcitenames=9,
    maxbibnames=9,
    abbreviate=false,
    eprint=false,
    backref,
    backrefstyle=none,
    doi=true
]{biblatex}
% allow line breaks in bibliogrpahy URLs, with penalty to avoid:
\setcounter{biburlnumpenalty}{9000} % numbers
\setcounter{biburllcpenalty}{9000}  % lower case
\setcounter{biburlucpenalty}{9000}  % upper case

\addbibresource{bibliography.bib}

% load hyperref as the last package to make it compatible with others!
\usepackage[pdftex,pdfpagelabels]{hyperref}
\hypersetup{
  pdfborder = {0 0 0},
  breaklinks = true,
  linktoc=all,
  pdfinfo={
    Author={\myshortauthor},
    Title={\mytitle}
    % NOTE fix date and time here for a reproducible build
    % CreationDate={D:20180424180424},
    % ModDate={D:20180424180424}
  }
}
\pdfinfoomitdate=1
\pdftrailerid{}
\pdfsuppressptexinfo15

% NEWCOMMANDS etc.

\newcommand{\twopartdef}[4]{\left \{ \begin{array}{ll}#1&#2\\#3&#4\end{array} \right .}
\newcommand{\stack}[2]{\begin{array}{c}#1\\#2\end{array}}
\newcommand{\ceiling}[1]{\left\lceil #1 \right\rceil}

% some logic
\let\lthen\rightarrow
\let\liff\leftrightarrow
\let\lxor\oplus

% will the real phi please stand up
\renewcommand{\phi}{\varphi}

\newcommand{\Nat}{\mathbb{N}}
\newcommand{\Pow}{\mathcal{P}}
\newcommand{\Lng}{\mathcal{L}}
\DeclareMathOperator{\quotient}{/}

% NOTE add your own \newcommand list here
